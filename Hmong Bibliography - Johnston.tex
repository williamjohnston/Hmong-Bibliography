\documentclass[11pt]{scrartcl}
%\usepackage[margin=1in]{geometry}
\input{D:/Synced/Drive/School/Research/"Dissertation Writing"/Preamble/Header.tex}
\KOMAoptions{paper=letter, parskip=half}
\input{D:/Synced/Drive/School/Research/"Dissertation Writing"/Preamble/FontsClassic.tex}
\input{D:/Synced/Drive/School/Research/"Dissertation Writing"/Preamble/Typesetting.tex}

%%% TITLES AND  HEADINGS %%%%%%%%%%%%%%%%%%%%%%%%%%%%%%%%%%%%%%%%%%%%
%1 = sec, 2 = subsec, 3 = subsubsec, 4 = para, 5 = subpara
\setcounter{secnumdepth}{3}%Numbering in-text; default = 2
\setcounter{tocdepth}{3}%Visibility in ToC; default = 2

\addtokomafont{disposition}{\rmfamily}
\RedeclareSectionCommands[font={\rmfamily\bfseries\LARGE}]{section}
\RedeclareSectionCommands[font={\rmfamily\bfseries\Large}]{subsection}
\RedeclareSectionCommands[font={\rmfamily\bfseries\normalsize}, runin=on, beforeskip=0pt,afterskip=0.5em]{paragraph,subparagraph}

%%% TABLES, LISTS, FIGURES, AND GRAPHICS %%%%%%%%%%%%%%%%%%%%%%%%%%%%
\usepackage[table]{xcolor}
\usepackage{enumitem}
\usepackage{graphicx}
	\graphicspath{ {../Images/} }
\usepackage[export]{adjustbox}	% Better environment controls

%%% LINGUISTICS %%%%%%%%%%%%%%%%%%%%%%%%%%%%%%%%%%%%%%%%%%%%%%%%%%%%%
\input{D:/Synced/Drive/School/Research/"Dissertation Writing"/Preamble/Linguistics.tex}

%%% BIBLIOGRAPHY SETUP %%%%%%%%%%%%%%%%%%%%%%%%%%%%%%%%%%%%%%%%%%%%%%
\usepackage{hyperref} %This will fail in beamer (which loads it by default)
\hypersetup{colorlinks=true, allcolors=cyan}

% Uses `unified.bst' from Semantics & Pragmatics
% style=authoryear is an okay alternative
\usepackage[natbib, backend=biber, style=unified, maxcitenames=3, maxbibnames=99, url=false, doi=false, isbn=false]{biblatex}

\renewcommand\postnotedelim{\addcomma\addspace}

\addbibresource{hmong.bib}
%\setlength\bibitemsep{0.5\baselineskip}


%%% CUSTOM COMMANDS AND ENVIRONMENTS %%%%%%%%%%%%%%%
\newcommand{\myrule}{\rule{.50\pagewidth}{.4pt}}
	%For generating uniform rules, in case it's separately needed
	
\newcommand{\frule}{
	\vspace{-1.5\baselineskip}
	\begin{center}\myrule\end{center}
	\vspace{-1\baselineskip}
	}
	%For generating uniform rules, in case it's separately needed

\newcommand{\currentref}{}% Dummy command for use in entry environment

\newenvironment{commoncore}%
	{	\vspace{-0.5\baselineskip}%
		\begin{center}% Introduces 1\baselineskip amount of white space before and after. 
		\begin{minipage}{.88\linewidth}%
		\KOMAoptions{parskip=half}%
		}
	{	\end{minipage}\medskip\par%
%		\myrule\medskip%
		\end{center}%
		}

%Or should I just make these real subsections/paragraphs? 
\newenvironment{entry}[1]%
	{	\renewcommand{\currentref}{#1}%
		\paragraph{\citetitle{\currentref}}\parencite{\currentref}\label{\currentref}%
		\begin{commoncore}%
		}
	{	\par%
%		\fullcite{\currentref}%
		\end{commoncore}%
		}%

%New environment for ``other works'' subsection at the end of each section
\newenvironment{otherworks}%
	{	\paragraph{Other works}%
		\begin{commoncore}%
		}
	{	\par%
		\end{commoncore}%
		}%

\newcommand{\rlink}[1]{\hyperref[#1]{\citealt{#1}}}%For ``see also'' lists
	
%%% METADATA %%%%%%%%%%%%%%%%%%%%%%%%%%%%%%%%%%%%%%%
\title{Hmong linguistics resources}
\author{William Johnston} 
\date{} % Activate to display a given date or no date (if empty), otherwise the current date is printed 

%%% HEADER AND FOOTER %%%%%%%%%%%%%%%%%%%%%%%%%%%%%%


%%% MAIN DOCUMENT %%%%%%%%%%%%%%%%%%%%%%%%%%%%%%%%%%
\begin{document}
\maketitle
\raggedbottom

\renewcommand{\abstractname}{}
\begin{abstract}
\noindent
This document is an annotated bibliography of published linguistic research on the Hmong language, primarily dealing with White Hmong and Green Mong varieties. As these two varieties have highly similar grammars, much of this research will be relevant to both (but please keep in mind that there may be some differences).

Please be aware that this list is not comprehensive. Many sources exist that are not included here. In particular, please note that: 

\begin{itemize}
\item These sources focus on the syntax and semantics (grammar and meaning) of Hmong, rather than on the language's sound system, its history, or its relationship to Hmong culture. 
\item Sources on other, related languages may be included when they are also relevant to Hmong, but those are not the main focus of this bibliography. 
\item This list is in-progress; further relevant works may continue to be added. 
\end{itemize}

The annotations in this document are meant to give an indication of the main contributions of each source and to point out those sources that are particularly useful and informative. However, be aware that these annotations represent my opinions, and that these are not necessarily shared by all linguists. 

For anyone interested in Hmong phonetics/phonology, historical linguistics, or sociolinguistics, or interested in Hmong-Mien languages more broadly, I suggest that you consult the Oxford Bibliographies page on Hmong-Mien linguistics 
(\href{https://www.oxfordbibliographies.com/view/document/obo-9780199772810/obo-9780199772810-0173.xml}{\textsc{web}}, 
\href{https://williamjohnston.github.io/files/Mortensen-2014.pdf}{\textsc{pdf}}) maintained by David Mortensen.
\end{abstract}

\newpage
\tableofcontents

%%%%%%%%%%%%%%%%%%%%%%%%%%%%%%%%%%%%%%%%%%%%%%%%%%%%%%%%
%	OVERVIEWS
%%%%%%%%%%%%%%%%%%%%%%%%%%%%%%%%%%%%%%%%%%%%%%%%%%%%%%%%
\newpage
\section{Grammars and Overviews}\label{overviews}
\vspace{-0.5\baselineskip}

Unfortunately there are no comprehensive, book-length grammars of Hmong---but there are still good references available. The two most recent sources mentioned here, \rlink{Jarkey2015} and \rlink{Mortensen2019}, are generally in agreement with one another, but have some slight differences. I recommend reading both of these sources even if you are only intersted in one particular variety.

\begin{entry}{Mottin1978}
This is a widely cited and foundational description of the White Hmong language, and among the most thorough. It describes many grammatical constructions and includes many, many examples. Though published in French, this is a useful resource for anyone interested in Hmong grammar. 
\end{entry}

\begin{entry}{Mortensen2019} 
A chapter-length description of Green Mong grammar, this is necessarily brief, but gives concise descriptions of most fundamental topics.
\end{entry}

\begin{entry}{Jarkey2015}
The initial chapter of \rlink{Jarkey2015} provides an overview of White Hmong grammar, briefly covering a wide range of fundamental topics.%
\end{entry}

%Yang, D. (1981). Notions sur la langue Hmong. (Unpublished).
%
%Yang, D. (1980). Dictionnaire Francaise-Hmong Blanc. Paris: Comite National d’Entraide.
%
%Lyman, T.A. (1979). Grammar of Mong Njua (Green Miao): A Descriptive Linguistic Study. Published by the Author.
%
%Purnell, H.C. ed. (1972). Miao and Yao Linguistic Studies: Selected Articles in Chinese, translated by C. Yu-hung and C. Kwo-Ray. Southeast Asia Program Data Paper No. 88. Ithaca, New York: Cornell University.
%
%Lyman, T.A. (1965). Excerpts from a Grammar of Green Miao and Green Miao Vocabulary. (Unpublished).

%%%%%%%%%%%%%%%%%%%%%%%%%%%%%%%%%%%%%%%%%%%%%%%%%%%%%%%%
%	VERBS
%%%%%%%%%%%%%%%%%%%%%%%%%%%%%%%%%%%%%%%%%%%%%%%%%%%%%%%%

\section{Verbs and the verb phrase}\vspace{-1\baselineskip}
%%%%%%%%%%%%%%%%%%%%%%%%%%%%%%%%%%%%%%%%%%%%%%%%%%%%%%%%
\subsection{Serial verb constructions} \label{serialverbconstructions}
%%%%%%%%%%%%%%%%%%%%%%%%%%%%%%%%%%%%%%%%%%%%%%%%%%%%%%%%

%\paragraph{Das verb im Chinesischen, Hmong, Vietnamesischen, Thai und Khmer: Vergleichende grammatik im rahmen der verbserialisierung, der grammatikalisierung und der attraktorpositionen \parencite{Bisang1992}.} %A book-length study of verb serialization and related phenomenon in the Sinosphere.
%\paragraph{Parataxis in White Hmong \parencite{Riddle1990b}.} 
%\paragraph{Parataxis as a target structure in Hmong \parencite{Riddle1990a}} %
%Li Harriehausen and Litton: Iconicity: A view from Green Hmong serial verbs

\begin{entry}{Jarkey2015}
{The most thorough reference on serial verb constructions in Hmong. Jarkey establishes a typology of four main types, discusses their properties in detail, and compares them with several superficially-similar constructions. (Includes an excellent chapter-length description of White Hmong grammar.)}
\end{entry}

\begin{entry}{Jarkey2010}
{Discusses serial verb constructions involving motion. The contents of this chapter are incorporated into \hyperref[Jarkey2015]{Jarkey 2015} (see Chapter 3, Section 1).} 
\end{entry}

\begin{entry}{Riddle1989}
Discusses several classes of serial verb construction in White Hmong (including ``instrumental'' constructions not distinguished by other authors) and argues that fine-grained semantic/pragmatic distinctions between certain verbs (e.g. \ol{muab} `take' vs.\ \ol{xuas} `grasp' vs.\ \ol{siv} `use' vs.\ \ol{tuav} `hold') determine whether a construction describes a single proposition or multiple propositions. Riddle's conclusions maybe be jeopardized: she appears to conflate multiple distinct classes of serial verb constructions (compare to \rlink{Jarkey2015}), meaning that the semantic/pragmatic contrasts she describes likely result from different underlying syntactic structures.
\end{entry}

\begin{otherworks}
Both \citet{Fuller1990} and \citet{Clark1992b} discuss the apparent link between serial verb constructions and other types of structures in which coordination, subordination, or modification are not overtly signaled. Clark non-standardly labels all of these phenomena as ``serialization'', while Riddle describes this trend as a preference for a ``paratactic surface target structure'' in Hmong, though neither paper analyzes this pattern in detail. 

%Fuller cites (a.o.): 
	%Goral, Donald. Verb concatenation in Southeast Asian languages: a cross-linguistic study. 
	%Hansell, Mark. Serial verbs and complement constructions in Mandarin: a clause-linkage analysis. 
	%Johnson, Charles, ed. (1981). Nkauj Ntsuab thiab Sis Nab, level 2. St. Paul: Macalester College.
	%Johnson, Charles. ed. (1981). Thawj tug tub ua qoob ua loo, level 1. St. Paul: Macalester College.
	%Li, Charles. (1988). Graamaticization in Hmong: verbs of saying. 21st International Conference on Sino-Tibetan Languages and Linguistics, Lund.
	%Li, Charles, Bettina Harriehausen, and Donald Litton. (1986). Iconicity: a view from Green Hmong serial verbs. Conference on Southeast Asia as a Linguistic Area, Chicago.
	%Lis, Nyiajpov. (1986). Lub neel daitaw. Australia: Roojntawv Neejahoob.
	%Ratliff. Martha. (1986b). The morphological functions of tone in White Hmong. Ph.D. diss., University of Chicago.
	%Strecker, David and Lopao Vang. (1986). White Hmong dialogues. Minneapolis: Southeast Asian Refugee Studies Project, University of Minnesota
	%Thoj, Cawv, trans. (1981). Kev tsim neej tshaib hauv Asaeslivkas. Washington, D. C.: Center for Applied Linguistics.
	%Xiong, Kong. (1980). Hmong-English phrasebook of daily language. Minneapolis: Kong Xiong.
	
\citealt{Jarkey1991} is a PhD dissertation updated and published in book form as \rlink{Jarkey2015}.
	
\citet{Harriehausen-Muhlbauer1992} discusses limited data from Green Mong in the context of natural language processing. The discussion of NLP is likely outdated, and more complete data can be found in \rlink{Jarkey2010} and \rlink{Jarkey2015}.

%In a conference presentation, \citet{Cooper-LeavittLonsdale2006} discuss two types of serial verb construction (discussed in greater detail in \rlink{Jarkey2015}).
\end{otherworks}

%Owensby, Laurel. 1986. Verb serialization in Hmong. In Glenn L. Hendricks, Bruce T. Downing & Amos S. Deinard (eds.), The Hmong in transition, 237-243. Staten Island, NY: Center for Migration Studies of New York; Minneapolis, MN: Southeast Asian Refugee Studies of the University of Minnesota.

%Riddle, E.M. (1990). “Parataxis in White Hmong.” Working Papers in Linguistics 39 (December 1990): 65-76.
%`read to my mother listen' 


%%%%%%%%%%%%%%%%%%%%%%%%%%%%%%%%%%%%%%%%%%%%%%%%%%%%%%%%
\subsection{Voice and valency}

\begin{entry}{CreswellSnyder2000}
{Describes two passive(-like) constructions in Hmong, those formed with \ol{raug} `hit' or \ol{mag} `trap', and those formed with \ol{yog} `to be'. Neither construction results in the demotion of the agent as expected in canonical passive constructions. \ol{Yog}-passives appear to be a copular construction, and \ol{raug}/\ol{mag}-passives appear to involve VP-embedding. (Data from White Hmong.)}
\end{entry}

%Fuller, J.W. (1985). “On the Passive Construction in Hmong.” Minnesota Papers in Linguistics and Philosophy of Language 10 (April 1985): 51-65.

%%%%%%%%%%%%%%%%%%%%%%%%%%%%%%%%%%%%%%%%%%%%%%%%%%%%%%%%
\subsection{Tense, aspect, and mood}

\begin{entry}{Li1991}
A fine-grained examination of several aspectual markers in Green Mong, including \ol{tau} (telic or ``attainment'' marker), \ol{lawm} (\textsc{perfect}), and \ol{taabtom} (\textsc{progressive}), some of which have multiple grammatical uses. Li also discusses the role of \ol{yuav} (\textsc{irrealis}). 
\end{entry}

%Thao, N. (2015). An exploration of Hmong lexical adverbs, aspects, and moods.  M.A. Thesis, California State University, Fresno.  

\begin{entry}{White2014}
A good overview of the distribution and use of various grammatical markers, including tense and aspect marking, aspectual verbs, mood/modality, certainty markers, and other adverbs. The descriptions of each word/morpheme are necessarily brief, and on some points appear to differ from those presented in other sources.
\end{entry}

%Ginsburg HO

%%%%%%%%%%%%%%%%%%%%%%%%%%%%%%%%%%%%%%%%%%%%%%%%%%%%%%%%
\subsection{Situational aspect/aktionsart}

\begin{entry}{Johnston2023}
Discusses Accomplishment verbs in White Hmong (\ol{nrhiav} `search for/find', \ol{noj} `eat', etc.), which are ``non-culminating'': they do not necessarily mean that their goal was reached. (For example, \ol{kuv \textbf{nrhiav} lub pob} `I searched for/found the ball' doesn't necessarily mean that the ball was found.) Focuses on verb serialization as a strategy for adding the ``culminating'' meaning. 
\end{entry}

%Jarkey, Nerida. 2004. Process and goal in White Hmong. In Tapp, Nicholas & Gary Yia Lee (eds.), The Hmong of Australia: Culture and diaspora, 175-189. Canberra: Pandanus Press.

%%%%%%%%%%%%%%%%%%%%%%%%%%%%%%%%%%%%%%%%%%%%%%%%%%%%%%%%
\subsection{Studies of specific verbs}

\begin{entry}{Taguchi2019}
A study of two motion verbs in Lan Hmyo, a West Hmongic language closely related to White Hmong/Green Mong. These verbs are \ol{luB} `come (home)' (Hmong \ol{los}/\ol{lus}), which Taguchi argues grammatically encodes the notion of ``home'', and \ol{\textipa{Da}A} `come' (Hmong \ol{tuaj}) which Taguchi claims does not.
\end{entry}

\begin{entry}{Enfield2003}
Many Southeast Asian languages have a single word that serves all of the following functions: (1) a verb meaning `get, acquire, attain', (2) an aspect marker associated with completion, (3) a possibility modal meaning `can, be able to', and (4) an introducer of ``descriptive complements''. Although this work focuses on data from Lao, it also surveys a variety of other languages and contains a significant amount of data on Hmong \ol{tau}. (More information on \ol{tau} can be found in \rlink{Li1991} and \rlink{Jarkey2015}.)	
\end{entry}



%Clark, M. (1979). Coverbs: Evidence for the Derivation of Prepositions from Verbs, New Evidence from Hmong. Working Papers in Linguistics, University of Hawaii at Manoa 11(2).
%
%Clark, M. (1979). Synchronically Derived Prepositions in Diachronic Perspective: Some Evidence from Hmong. Unpublished paper presented at the 12th International Conference on Sino-Tibetan Languages and Linguistics, Paris, October 19-21.
%
%Clark, Marybeth. 1982. Some auxiliary verbs in Hmong. In Bruce T. Downing \& Douglas P. Olney (eds.), The Hmong in the West, 125-141. Minneapolis: Southeast Asian Refugee Studies Project, Center for Urban and Regional Affairs, University of Minnesota.
%
%Clark, M. (1980). Source Phrases in White Hmong (Laos). Working Papers in Linguistics, University of Hawaii at Manoa 12(2): 1-49.
%
%Clark, M. (1980). Derivation between Goal and Source Verbs in Hmong. Working Papers in Linguistics, University of Hawaii at Manoa 12(2): 51-59.
%
%Jaisser, A. (1990). “Delivering an Introduction to Psycho-Collocations with SIAB in White Hmong.” Linguistics of the Tibeto-Burman Area 13 (Spring 1990): 159-178.
%


%%%%%%%%%%%%%%%%%%%%%%%%%%%%%%%%%%%%%%%%%%%%%%%%%%%%%%%%
\section{Nouns and the noun phrase}
%\subsection{Nouns}

% Ratliff 1991 (The development of nominal/non-nominal tone marking in Shimen Hmong'?

%%%%%%%%%%%%%%%%%%%%%%%%%%%%%%%%%%%%%%%%%%%%%%%%%%%%%%%%
\subsection{Classifiers}
%%%%%%%%%%%%%%%%%%%%%%%%%%%%%%%%%%%%%%%%%%%%%%%%%%%%%%%%

\begin{entry}{SakuragiFuller2013}
{A study examining the factors that affect Hmong speakers' choice of classifiers. The results suggest that classifiers are associated with both particular shapes and particular functions. (E.g., \ol{tus} and \ol{txoj} can both be used for nouns that describe long, thin objects, but \ol{tus} is preferred over \ol{txoj} when the normal use of that noun involves grasping it.) In some cases, Hmong speakers can be led to prefer different classifiers for the same noun, depending on whether they focus on the shape of the noun in question, or on its function.} 
\end{entry}

%\begin{entry}{SimpsonEtAl2011}
%\end{entry}

\begin{entry}{Bisang1993} 
On the basis of syntactic and semantic tests, Bisang shows that what are usually called ``classifiers'' in Hmong are in fact a mixed bag of true classifiers, quantifiers, measure words, and class nouns. This is a detailed and fine-grained description, which offers an contrasting view of the ``double classifier constraint'' discussed by \rlink{Ratliff1991} and an explanation for the ``referential salience'' analysis of \rlink{Riddle1989a}.
\end{entry}

%\begin{entry}{Riddle1989a}
%\end{entry}

\begin{entry}{Ratliff1991}
{Discusses cases of ``double classifiers'' in Hmong. These usually involve the plural classifier \ol{cov} being added to a classifier-noun pair, but require the noun to be semantically underspecified (e.g. \ol{cov $+$ phau ntawv}, `the books'). Ratliff argues that the second classifier acts as a noun in these cases, forming the first part of a compound word, and relates this to a broader pattern of syntactic flexibility in Hmong. (Data from White Hmong.)} 
\end{entry}

%Melton, J.B. (1991). An Analysis of Hmong Noun Classifiers. Master’s Thesis, San Diego State University.

%Jaisser, A. (1987). “Hmong Classifiers: A Problem Set.” Linguistics of the Tibeto-Burman Area 10 (Fall 1987): 169-176.

%Donnelly, N.D. (1982). Preliminary Study of Noun Classifier Use in the Hmong Language. Department of
%Anthropology, University of Washington (Unpublished).

%Nathan M. White: Classifiers in Hmong

%Classifiers in East and Southeast Asian languages: Counting and beyond \parencite{Bisang1999}. A study of noun classifiers in East and Southeast Asian languages, including Hmong.

%Gerner, M. & W. Bisang (2009). “Inflectional Classifiers in Weining Ahmao: Mirror of the History of a people.” Folia Linguistica Historica 30(1/2), 183-218. Societas Linguistica Europaea.
%


%%%%%%%%%%%%%%%%%%%%%%%%%%%%%%%%%%%%%%%%%%%%%%%%%%%%%%%%
\subsection{Demonstratives}

%Gerner, M. (2009). “Deictic features of demonstratives: A typological survey with special reference to the Miao group.” Canadian Journal of Linguistics, 54(1), 43-90.
%

\begin{entry}{Ratliff1997}
{Discusses the White Hmong demonstrative \ol{ko} `that (near you)', which has been omitted from several other accounts. Ratliff situates \ol{ko} within the full demonstrative system of White Hmong. This type of system is unusual among Southeast Asian languages, and the historical development of this system is discussed.}
\end{entry}

%%%%%%%%%%%%%%%%%%%%%%%%%%%%%%%%%%%%%%%%%%%%%%%%%%%%%%%%
\subsection{Pronouns and Binding}

%Bansal MS

\begin{entry}{Mortensen2004}
Discusses A and A$'$ binding, including anaphoric binding into proper names, full pronominals (e.g.\ \ol{nwg}, \textsc{3sg}), kinship pronominals (e.g.\ \ol{yawg}, `male relative'), null \emph{pro}, and \ol{tug kheej} `self' forms. Describes an apparent ``competition'' between these forms. (Data from Green Mong.)
\end{entry}

\begin{entry}{Ratliff1992b}
Provides data on so-called ``expansion pronouns'' in White Hmong (sometimes referred to as ``associative'' forms in other languages). These forms combine an NP with a pronoun, in order to describe a larger group containing the noun. For example, \textit{Nplias nkawd} (= Nplias \textsc{2du}) describes a group of two people, of whom Nplias is one. Ratliff explores two possible analyses.
\end{entry}

%Guthrie, S. (1981). Some Observations on Language Universals and Promononialization in English and Hmong.
%(Unpublished).
%


%%%%%%%%%%%%%%%%%%%%%%%%%%%%%%%%%%%%%%%%%%%%%%%%%%%%%%%%
\section{Clause and sentence structure}

%%%%%%%%%%%%%%%%%%%%%%%%%%%%%%%%%%%%%%%%%%%%%%%%%%%%%%%%
%\subsection{Word Order}\label{wordorder}

%Sposato, A. (2014). "Word order in Miao-Yao (Hmong-Mien)." Linguistic Typology 18(1): 83-140.
%


%%%%%%%%%%%%%%%%%%%%%%%%%%%%%%%%%%%%%%%%%%%%%%%%%%%%%%%%
\subsection{Complementizers and Complement Clauses}\label{complementation}

\begin{entry}{Jarkey2006}
Discusses several distinct types of complement clause in Hmong, as well as the verbs that introduce them. The relationship between clause type and choice of complementizer (\ol{(hais) tias}, \ol{kom}, \ol{tias kom}, and the null complementizer) is described in detail.
\end{entry}

%Jaisser, A.C. (1986). “The Morpheme ‘Kom’: A First Analysis and Look at Embedding in Hmong.” In The Hmong in Transition, ed. by G.L. Hendricks, B.T. Downing, and A.S. Deinard, 245-260.

%\paragraph{Complementation in Hmong \parencite{Jaisser1984}.} %A master’s thesis exploring the structure of complement clauses in Hmong.

%%%%%%%%%%%%%%%%%%%%%%%%%%%%%%%%%%%%%%%%%%%%%%%%%%%%%%%%
\subsection{Conjunctions and Discourse Particles}

%\paragraph{Discourse functions of particle \ol{tes} in Green Hmong \parencite{DejAmorn2004}.} %(Is this actually the last name? Or is it the first name?)
%Describes ways in which the Mong Leng (Green Mong) particle tes serves to connect discourse using primary data from a recorded narrative.

%Lyman, T.A. The Particle ‘le’ in Green Miao. Asia Aakhanee: Southeast Asian Survey 1(1): 1-14.

\begin{entry}{Li1989}
Argues that the Green Mong clausal conjunctions \ol{huas} and \ol{hab} function as switch-reference markers: \ol{huas} conveys a weak contrast and is used in different-subject contexts, while the semantically-neutral \ol{hab} is used in same-subject contexts. Discusses possible historical origins of this behavior.
\end{entry}

\begin{otherworks}
\citet{Clark1988,Clark1992} argues that the White Hmong conjunction \ol{los} has an inchoative meaning, and that this allows conjunctions to function as topicalizers. This claim is not presented formally. Also touches on other clause linkers: \ol{mas}, \ol{ho}, \ol{ces}, and \ol{thiab}.

\citet{Bleske2003} discusses eight ``particles'' in White Hmong/Green Mong that serve as conjunctions, clause-linkers, or discourse-related adverbs (\ol{ces}, \ol{hos}/\ol{huas}, \ol{thiab}/\ol{hab}, \ol{ho}, \ol{kuj}, \ol{mam}, \ol{mas}, and \ol{ma}) and describes their usage based on their occurrence in four texts. Largely corroborates other sources.

\citet{Riddle1992,Riddle1993} presents data on the discourse function of the White Hmong relative clause marker \ol{uas}, which is argued to specify or restrict the reference of the relative clause. Does not discuss the status of \ol{uas} as a complementizer.
\end{otherworks}



%%%%%%%%%%%%%%%%%%%%%%%%%%%%%%%%%%%%%%%%%%%%%%%%%%%%%%%%
%\subsection{Topic Prominence and Topicalization}\label{topics}

%Birnschein, K.A. (2019). A Text-Based Exploration of Topics in White Hmong Grammar. MA Thesis, University of North Dakota. 
%
%Fuller, J.W. (1985). “Zero Anaphora and Topic Prominence in Hmong.” In the Hmong in Transition, ed. by G.L.
%Hendricks, B.T. Downing, and A.S. Deinard, 261-277.
%

%\paragraph{Topic markers in Hmong \parencite{Fuller1987}.} %Describes the particles used to mark topics in Hmong Daw (White Hmong).

%\paragraph{Topic and comment in Hmong \parencite{Fuller1985}.} %A dissertation arguing that Hmong syntax is best viewed as topic-prominent rather than subject-prominent.

%\begin{otherworks}
%\rlink{Ratliff1992b}, \rlink{Clark1992}, and \rlink{Clark1988} in the section on Conjunction and Complementation, all of which touch on topicalization to an extent.
%\end{otherworks} 

%%%%%%%%%%%%%%%%%%%%%%%%%%%%%%%%%%%%%%%%%%%%%%%%%%%%%%%%
%\subsection{Relative Clauses}

%%%%%%%%%%%%%%%%%%%%%%%%%%%%%%%%%%%%%%%%%%%%%%%%%%%%%%%%
\subsection{Questions}

\begin{entry}{Clark1985}
{Discusses the ``V-not-V'' strategy used to form yes-no questions in many Southeast Asian languages. For example, in Hmong, \ol{koj mus (los) tsis mus} (lit.\ ``You go (or) not go?'') can mean ``Are you going?'' Ten languages are studied, and among these, Hmong shows two uncommon features: it makes heavier use of ``V-not-V'' questions than the other languages, and in yes-no questions, the question word \ol{puas} precedes the verb.}
\end{entry}

%%%%%%%%%%%%%%%%%%%%%%%%%%%%%%%%%%%%%%%%%%%%%%%%%%%%%%%%
%\section{Morphology}

%Ratliff, M. (2010). Meaningful Tone: A Study of Tonal Morphology in Compounds, Form Classes and Expressive
%Phases in White Hmong. Dekalb, IL: Northern Illinois Press.
%
%Ratliff, M. (1992). Meaningful Tone: A Study of Tonal Morphology In Compounds, Form Classes, and Expressive
%Phrases in White Hmong. Dekalb: Center for Southeast Asian Studies, Northern Illinois University.
%
%Ratliff, M. (1987). “Tone Sandhi Compounding in White Hmong.” Linguistics of the Tibeto-Burman Area 10 (Fall 1987): 71-105.
%
%\paragraph{An analysis of some tonally differentiated doublets in White Hmong (Miao) \parencite{Ratliff1986}.} %An examination of morphologically related words that differ only in tone in Hmong Daw (White Hmong).

%%%%%%%%%%%%%%%%%%%%%%%%%%%%%%%%%%%%%%%%%%%%%%%%%%%%%%%%
%\section{Semantics}

%\paragraph{Deixis and anaphora and prelinguistic universals \parencite{Clark2000}.} %An examination of the three-way proximal-medial-distal deictic contrast in Hmong and Vietnamese coupled with an argument that this same deictic contrast is “prelinguistic” and is found in wolf and chimpanzee behavior.

%\paragraph{Process and goal in White Hmong \parencite{Jarkey2004}. } %Compares the expression of telic and atelic propositions in Hmong Daw (White Hmong) and English.

%\paragraph{Metaphorically speaking in White Hmong \parencite{Riddle1999}.} %An exploration of metaphorical language in Hmong Daw (White Hmong).

%Nathan M. White: Class properties of modality-marking words in White Hmong

%%%%%%%%%%%%%%%%%%%%%%%%%%%%%%%%%%%%%%%%%%%%%%%%%%%%%%%
%\section{Historial linguistics}

%Ratliff, M. 2010. Hmong-Mien Language History. Canberra: The Australian National University, Pacific Linguistics.

%White, N.W.. (2021). "Prehistory of Verbal Markers in Hmong: What Can We Say?" Studia Linguistica 75(2)  345–374.

%%%%%%%%%%%%%%%%%%%%%%%%%%%%%%%%%%%%%%%%%%%%%%%%%%%%%%%%
%\subsection{Miscellaneous}

%Lyman, T.A. (1987). “The Word Nzi in Green Hmong (Miao).” Linguistics of the Tibeto-Burman Area 10 (Fall 1987):
%142-143.

%Pederson, E.W. (1985). Intensive and Expressive Language in White Hmong (Hmoob Dawb). Master’s Thesis,
%University of California, Berkeley.

%Clark, Marybeth & Amara Prasithrathsint. 1985. Synchronic lexical derivation in Southeast Asian languages. In Ratanakul, Suriya, et al. (eds.), Southeast Asian Linguistic Studies Presented to André-G. Haudricourt, 34-81. Institute of Language and Culture for Rural Development, Mahidol University.

%Bisang, Walter. 1996. Areal typology and grammaticalization: Processes of grammaticalization based on nouns and verbs in East and mainland South East Asian languages. Studies in Language 20(3). 519-597.

%Nathan M. White: Quantifier float in Hmong (talk at Academia Sinica)

%Nathan White: ‘Affixation in an isolating language? Wordhood and the case of Hmong’ (talk at SEALS 28)

%Nathan White (2020): `Word in Hmong' 

%Bruhn (2007) LF Wh-Movement in Mong Leng: this is course final paper, not published
	%island sensitivity

%Ratliff, Martha. 1986. Two-word expressives in White Hmong. In Glenn L. Hendricks, Bruce T. Downing & Amos S. Deinard (eds.), The Hmong in transition, 219-236. Staten Island, NY: Center for Migration Studies of New York; Minneapolis, MN: Southeast Asian Refugee Studies of the University of Minnesota.

%%%%%%%%%%%%%%%%%%%%%%%%%%%%%%%%%%%%%%%%%%%%%%%%%%%%%%
\newpage
\printbibliography[title={Full References}]
\end{document}

%The Hmong Language
%Compiled by Mark E. Pfeifer, PhD
%Last accessed: 4/2/23
%
%Go back and double-check this. Also get corpora, phonetics/phonology, morphology, sociolinguistics sources eventually. 
